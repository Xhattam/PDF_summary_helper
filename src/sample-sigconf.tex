%%%% Proceedings format for most of ACM conferences (with the exceptions listed below) and all ICPS volumes.
% * <ewp123@ums.ac.id> 2018-08-27T17:16:36.668Z:
%
% ^.
%\documentclass[sigconf]{acmart}
%%%% As of March 2017, [siggraph] is no longer used. Please use sigconf (above) for SIGGRAPH conferences.

%%%% Proceedings format for SIGPLAN conferences 
\documentclass[sigconf]{acmart}

%%%% Proceedings format for SIGCHI conferences
% \documentclass[sigchi, review]{acmart}

%%%% To use the SIGCHI extended abstract template, please visit
% https://www.overleaf.com/read/zzzfqvkmrfzn


\usepackage{booktabs} % For formal tables
\usepackage{graphicx}
\usepackage{url} 


% Copyright
%\setcopyright{none}
%\setcopyright{acmcopyright}
%\setcopyright{acmlicensed}
\setcopyright{rightsretained}
%\setcopyright{usgov}
%\setcopyright{usgovmixed}
%\setcopyright{cagov}
%\setcopyright{cagovmixed}


% DOI
%\acmDOI{10.475/123_4}

% ISBN
%\acmISBN{123-4567-24-567/08/06}

%Conference
\acmConference[RDSM, CIKM'18]{2nd International Workshop on Rumours and Deception in Social Media (RDSM) 
co-located with International Conference on Information and Knowledge Management}{October 2018}{
 Torino, Italy}
\acmYear{2018}
\copyrightyear{2018}


%\acmArticle{4}
%\acmPrice{15.00}

% These commands are optional
%\acmBooktitle{Transactions of the ACM Woodstock conference}
%\editor{Jennifer B. Sartor}
%\editor{Theo D'Hondt}
%\editor{Wolfgang De Meuter}


\begin{document}
\title{Stance Classification for Rumour Analysis in Twitter: \\ Exploiting Affective Information and Conversation Structure}
%\titlenote{Produces the permission block, and
%  copyright information}
%\subtitle{Extended Abstract}
%\subtitlenote{The full version of the author's guide is available as
%  \texttt{acmart.pdf} document}


\author{Endang Wahyu Pamungkas}
%\authornote{Dr.~Trovato insisted his name be first.}
%\orcid{1234-5678-9012}
\affiliation{%
  \institution{Dipartimento di Informatica, University of Turin}
  \streetaddress{Corso Svizzera, 185}
  \city{Turin}
  \state{Italy}
  \postcode{10149}
}
\email{pamungka@di.unito.it}

\author{Valerio Basile}
\affiliation{%
  \institution{Dipartimento di Informatica, University of Turin}
  \streetaddress{Corso Svizzera, 185}
  \city{Turin}
  \state{Italy}
  \postcode{10149}
}
\email{basile@di.unito.it}

\author{Viviana Patti}
\affiliation{%
  \institution{Dipartimento di Informatica, University of Turin}
  \streetaddress{Corso Svizzera, 185}
  \city{Turin}
  \state{Italy}
  \postcode{10149}
}
\email{patti@di.unito.it}


\begin{abstract}
%Resolving rumour 
Analysing how people react to rumours associated with news in social media is an important task to prevent the spreading of misinformation, which is nowadays widely recognized as a dangerous tendency. 
%, since social media is full of false claims. 
In social media conversations, users show different stances and attitudes towards rumourous stories. Some users take a definite stance, supporting or denying the rumour at issue, while others just comment it, or ask for additional evidence related to the veracity of the rumour. On this line, a new 
%stance classification 
shared task has been proposed at SemEval-2017 (Task 8, SubTask A), which is focused on rumour stance classification in English tweets. The goal is predicting user stance towards emerging rumours in Twitter, in terms of supporting, denying, querying, or commenting the original rumour, looking at the conversation threads originated by the rumour. This paper describes a new approach to this task, where the use of conversation-based and affective-based features, covering different facets of affect, has been explored. 
%our effort in improving the system's performance to classify stance in SemEval-2017 Task 8 SubTask A. 
%Feature engineering becomes our main contribution in %this work. 
%We propose two groups of feature including %conversational-based and affective-based features. 
Our classification model outperforms the best-performing systems for stance classification at SemEval-2017 Task 8,
%and the state of the art, 
showing the effectiveness of the feature set proposed.
\end{abstract}

%
% The code below should be generated by the tool at
% http://dl.acm.org/ccs.cfm
% Please copy and paste the code instead of the example below.
%
\begin{CCSXML}
<ccs2012>
<concept>
<concept_id>10002951.10003317.10003347.10003353</concept_id>
<concept_desc>Information systems~Sentiment analysis</concept_desc>
<concept_significance>500</concept_significance>
</concept>
<concept>
<concept_id>10010147.10010178.10010179</concept_id>
<concept_desc>Computing methodologies~Natural language processing</concept_desc>
<concept_significance>500</concept_significance>
</concept>
</ccs2012>
\end{CCSXML}

\ccsdesc[500]{Information systems~Sentiment analysis}
\ccsdesc[500]{Computing methodologies~Natural language processing}


\keywords{Affect, Rumour Analysis, Sentiment Analysis, Stance Detection}


\maketitle

%\input{samplebody-conf}


\section{Introduction}

\section{SemEval-2017 Task 8: RumourEval} \label{sec:dataset}


\begin{table}
\begin{center}
\begin{tabular}{ p{3.1cm} p{0.6cm}p{0.6cm}p{0.6cm}p{0.6cm}  }
 \hline
 \multicolumn{5}{c}{\textbf{Development Data}} \\
 \hline
   \textbf{Rumour} & \textbf{S} & \textbf{D} & \textbf{Q} & \textbf{C}\\
 \hline
 Germanwings & 69 & 11 & 28 & 173\\
 \hline
 \hline
 \multicolumn{5}{c}{\textbf{Training Data}} \\
 \hline
   \textbf{Rumour} & \textbf{S} & \textbf{D} & \textbf{Q} & \textbf{C}\\
 \hline
 Charlie Hebdo & 239 & 58 & 53 & 721\\
 Ebola Essien &  6 & 6 & 1 & 21\\
 Ferguson &  176 & 91 & 99 & 718\\
 Ottawa Shooting &  161 & 76 & 63 & 477\\
 Prince Toronto &  21 & 7 & 11 & 64\\
 Putin Missing &  18 & 6 & 5 & 33\\
 Sydney Siege &  220 & 89 & 98 & 700\\
 \hline
 \textbf{Total} & 841 & 333 & 330 & 2734\\
 \hline
 \hline
 \multicolumn{5}{c}{\textbf{Testing Data}} \\
 \hline
   \textbf{Rumour} & \textbf{S} & \textbf{D} & \textbf{Q} & \textbf{C}\\
 \hline
 Ferguson &  15 & 4 & 17 & 66\\
 Ottawa Shooting &  10 & 2 & 20 & 91\\
 Sydney Siege &  5 & 1 & 12 & 69\\
 Charlie Hebdo &  9 & 2 & 8 & 74\\
 Germanwings &  11 & 5 & 15 & 71\\
 Marina Joyce &  5 & 30 & 10 & 110\\
 Hillary's Illness &  39 & 27 & 24 & 297\\
\hline
 \textbf{Total} & 94 & 71 & 106 & 778\\
\hline
\end{tabular}
\end{center}
\caption{\label{data-distribution} Semeval-2017 Task 8 (A) dataset distribution.}
\end{table}

\section{Proposed Method}

\subsection{Structural Features}


\begin{itemize}
\item[]{\bf Retweet Count}: The number of retweet of each tweet.
\item[]{\bf Question Mark}: presence of question mark "?"; binary value (0 and 1).
\item[]{\bf Question Mark Count}: 
%Different with the previous feature that has binary value (0 and 1), 
%the value of this feature is the 
number of question marks present in the tweet.
\item[]{\bf Hashtag Presence}: this feature has a binary value 0 (if there is no hashtag in the tweet) or 1 (if there is at least one hashtag in the tweet). 
\item[]{\bf Text Length}: number of characters after removing Twitter markers such as hashtags, mentions, and URLs.
\item[]{\bf URL Count}: number of URL links in the tweet.
\end{itemize}

\subsection{Conversation Based Features}


\begin{itemize}
\item[]{\bf Text Similarity to Source Tweet}: Jaccard Similarity of each tweet with its source tweet.
\item[]{\bf Text Similarity to Replied Tweet}: the degree of similarity between the tweet with the previous tweet in the thread (the tweet is a reply to that tweet).
\item[]{\bf Tweet Depth}: the depth value is obtained by counting the node from sources (roots) to each tweet in their hierarchy.
\end{itemize}

\subsection{Affective Based Features}


\begin{itemize}
\item[]{\bf Emolex}: it contains 14,182 words associated with eight primary emotion based on the Plutchik model \cite{mohammad2013crowdsourcing,plutchik2001nature}.\item[]{\bf EmoSenticNet(EmoSN)}: it is an enriched version of SenticNet \cite{cambria2014senticnet} including 13,189 words labeled by six Ekman's basic emotion \cite{poria2013enhanced,ekman1992argument}.
\item[]{\bf Dictionary of Affect in Language (DAL)}: includes 8,742 English words labeled by three scores representing three dimensions: Pleasantness, Activation and Imagery \cite{whissell2009using}.
\item[]{\bf Affective Norms for English Words (ANEW)}: 
%it was developed by  and c
consists of 1,034 English words \cite{bradley1999affective} rated with ratings based on the 
Valence-Arousal-Dominance (VAD) 
model \cite{osgood_measurement_1957}.
\item[]{\bf Linguistic Inquiry and Word Count (LIWC)}: this psycholinguistic resource \cite{pennebaker2001linguistic} includes 4,500 words distributed into 64 emotional categories including positive (PosEMO) and negative (NegEMO). 
\end{itemize}

\subsection{Dialogue-Act Features}


\begin{itemize}
\item[]{\bf Agree-accept:} Assent, Certain, Affect;
\item[]{\bf Reject:} Negate, Inhib;
\item[]{\bf Info-request:} You, Cause;
\item[]{\bf Opinion:} Future, Sad, Insight, Cogmech.
\end{itemize}

\section{Experiments, Evaluation and Analysis}


\begin{table}
\begin{center}
\begin{tabular}{ p{0.5cm}p{4cm}p{1.5cm}  }
 \hline
   \textbf{No.} & \textbf{Systems} & \textbf{Accuracy} \\
 \hline
 1. & Turing's System & 78.4 \\
 2. & Aker et al. System & 79.02 \\
 3. & Our System & \textbf{79.5} \\
 \hline
 & RumourEval Baseline & 74.1\\
 \hline
\end{tabular}
\end{center}
\caption{\label{performance-comparison} Results and comparison with state of the art}
\end{table}


\begin{table}
\begin{center}
\begin{tabular}{ p{2cm} p{0.75cm} p{0.75cm} p{0.75cm} p{0.75cm} }
 \hline
   & \textbf{S} & \textbf{D} & \textbf{Q} & \textbf{C}\\
 \hline
 \textbf{Support} &  \textbf{27} & 0 & 3 & 64\\
 \hline
 \textbf{Deny} &  2 & \textbf{0} & 1 & 68\\
 \hline
 \textbf{Query} & 0 & 0 & \textbf{50} & 56\\
 \hline
 \textbf{Comment} & 13 & 0 & 8 & \textbf{757}\\
 \hline
\end{tabular}
\end{center}
\caption{\label{confusion-matrix} Confusion Matrix}
\end{table}


\begin{table}
\begin{center}
\begin{tabular}{ p{2cm} p{0.75cm} p{0.75cm} p{0.75cm} p{0.75cm} }
 \hline
   & \textbf{S} & \textbf{D} & \textbf{Q} & \textbf{C}\\
 \hline
 \textbf{Support} &  \textbf{39} & 14 & 5 & 13\\
 \hline
 \textbf{Deny} &  8 & \textbf{28} & 5 & 30\\
 \hline
 \textbf{Query} & 2 & 3 & \textbf{62} & 4\\
 \hline
 \textbf{Comment} & 14 & 14 & 2 & \textbf{41}\\
 \hline
\end{tabular}
\end{center}
\caption{\label{confusion-matrix-balance} Confusion Matrix on Balanced Dataset}
\end{table}

\subsection{\em Error analysis}

\section{Conclusion}

\bibliographystyle{ACM-Reference-Format}

\bibliography{sample-bibliography}

\end{document}