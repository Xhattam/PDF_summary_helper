\documentclass[11pt,a4paper]{article}
\usepackage[hyperref]{emnlp2018}
%\hypersetup{draft}

\usepackage{times}
\usepackage{latexsym}
\usepackage{pbox}
\usepackage{url}
\aclfinalcopy % Uncomment this line for the final submission

%\setlength\titlebox{5cm}

\newcommand\BibTeX{B{\sc ib}\TeX}
\newcommand\confname{EMNLP 2018}
\newcommand\conforg{SIGDAT}

\usepackage{amsmath,amsfonts,amsthm}
\usepackage{color}
\usepackage{graphicx}
\usepackage{pifont}% http://ctan.org/pkg/pifont
\newcommand{\cmark}{\ding{51}}%
\newcommand{\xmark}{\ding{55}}%
\usepackage{footnote}
\makesavenoteenv{tabular}
\makesavenoteenv{table}

\newtheoremstyle{mydef}%
{}{}{\normalfont}{}{\itshape}{.\,}{ }{}
\theoremstyle{mydef}
\newtheorem{definition}{Definition}

\newtheoremstyle{myprob}%
{}{}{\normalfont}{}{\scshape}{.\,}{ }{}
\theoremstyle{myprob}
\newtheorem{problem}{Problem}

\newtheoremstyle{mylayer}{}{}{\normalfont}{}{\itshape}{.\,}{ }{}\theoremstyle{mylayer}
\newtheorem{layer1}{}

%%%%%%%%%%%%%%%

\begin{document}

\title{Multi-style Generative Reading Comprehension}

\author{Kyosuke Nishida$^1$, 
Itsumi Saito$^1$, 
Kosuke Nishida$^1$, 
\\{\bf Kazutoshi Shinoda}$^2$\thanks{\ \ Work done during an internship at NTT.}, 
{\bf Atsushi Otsuka}$^1$,
{\bf Hisako Asano}$^1$, 
{\bf Junji Tomita}$^1$\\
  $^1$NTT Media Intelligence Laboratory, NTT Corporation \hspace{1.5em}  $^2$The University of Tokyo\\
  {\tt kyosuke.nishida@acm.org}
}

\date{}

\maketitle
\begin{abstract}
This study focuses on the task of multi-passage reading comprehension (RC) where an answer is provided in natural language. Current mainstream approaches treat RC by extracting the answer span from the provided passages and cannot generate an abstractive summary from the given question and passages. Moreover, they cannot utilize and control different styles of answers, such as concise phrases and well-formed sentences, within a model. 
In this study, we propose a style-controllable Multi-source Abstractive Summarization model for QUEstion answering, called Masque. The model is an end-to-end deep neural network that can generate answers conditioned on a given style. 
Experiments with MS MARCO 2.1 show that our model achieved state-of-the-art performance %in terms of Rouge-L 
on two tasks with different answer styles.
\end{abstract}

\section{Introduction}
\label{sec:intro}

Here, current mainstream studies have treated RC as a process of extracting an answer span from one passage~\citep{RajpurkarZLL16,RajpurkarJL18} or multiple passages~\citep{JoshiCWZ17}, which is usually done by predicting the start and end positions of the answer~\citep{Yu18,DevlinCLT18}.

\begin{figure}[t!]
\centering
\includegraphics[width=.47\textwidth]{./images/masque_mixratio2.eps}
\caption{Visualization of how our model generates the answer. Given a style (\textbf{Top}: well-formed NLG, \textbf{Bottom}: concise Q\&A), our model chooses to generate words from a fixed vocabulary or copy words from the question and multiple passages at each decoding step.
}
\label{fig:mixratio}
\end{figure}
 
The methods used in previous studies cannot utilize and control different answer styles within a model.


\section{Problem Formulation}
\label{sec:problem}

The task considered in this paper, is defined as:
\begin{problem}
\label{prob:prob}
Given a question with $J$ words $x^q = \{x^q_1, \ldots, x^q_J\}$, a set of $K$ passages, where each $k$-th passage is composed of $L$ words $x^{p_k} = \{x^{p_k}_1, \ldots, x^{p_k}_{L}\}$, and an answer style $s$, an RC system %distinguishes whether the question is answered on the basis of the provided passages and 
outputs an answer $y = \{y_1, \ldots, y_T \}$ conditioned on the style.
%when the question can be answered.
\end{problem}

\section{Proposed Model} 

Our proposed model, \textit{Masque}, is based on \textbf{multi-source abstractive summarization}.



\begin{figure}[t!]
\centering
\includegraphics[width=.48\textwidth]{./images/Masque_model11.eps}
\caption{The Masque model architecture.}
\label{fig:model}
\end{figure}

Masque directly models 
the conditional probability $p(y|x^q, \{x^{p_k}\}, s)$.
%the conditional probabilities, $p(y|x^q, \{x^{p_k}\}, s)$ and $p(a|x^q, \{x^{p_k}\}, s)$. 
Figure~\ref{fig:model} shows the model architecture. It consists of the following modules.
\begin{layer1}
The \textbf{question-passages reader} (\S\ref{sec:reader}) models interactions between the question and passages.
\end{layer1}
\begin{layer1}
The \textbf{passage ranker} (\S\ref{sec:ranker}) finds relevant passages to the question.
\end{layer1}

\subsection{Question-Passages Reader}
\label{sec:reader}

Given a question and  passages, the question-passages reader matches them so that the interactions among the question (passage) words conditioned on the passages (question) can be captured.

\subsubsection{Word Embedding Layer}

A pre-trained weight matrix $W^e \in \mathbb{R}^{d_\mathrm{word} \times V}$ such as GloVe~\citep{PenningtonSM14}. Next, it uses contextualized word representations, ELMo~\citep{PetersNIGCLZ18}, which is a character-level two-layer bidirectional language model pre-trained on a large-scale corpus. 

\subsubsection{Dual Attention Layer}
\label{sec:dual}

This layer fuses information from the passages to the question as well as from the question to the passages in a dual mechanism. 

It first computes a similarity matrix $U^{p_k} \in \mathbb{R}^{L{\times}J}$ between the question and $k$-th passage, as is done in \citep{SeoKFH17}, where
\begin{align}
U^{p_k}_{lj} = {w^a}^\top [ E^{p_k}_l; E^q_j; E^{p_k}_l \odot E^q_j ]
\end{align}
indicates the similarity between the $l$-th word of the $k$-th passage and the $j$-th question word. 

\subsubsection{Modeling Encoder Layer}

This layer uses a stack of Transformer encoder blocks for question representations and obtains $M^q \in \mathbb{R}^{d \times J}$ from $G^{p \rightarrow q}$. 

\paragraph{Copy distribution.}

The layer takes $s_t$ as the query and outputs $\alpha^q_t \in \mathbb{R}^J$ ($\alpha^p_t \in \mathbb{R}^{KL}$) as the attention weights and $c^q_t \in \mathbb{R}^d$ ($c^p_t \in \mathbb{R}^d$) as the context vectors for the question (passages):
\begin{align}
e^q_j &= {w^q}^\top \tanh(W^{qm} M_j^q + W^{qs} s_t +b^q), \\
\alpha^q_t &= \mathrm{softmax}(e^q), \\ 
c^q_t &= \textstyle \sum_j \alpha^q_{tj} M_j^q, \\
%\end{align}
%\begin{align}
e^{p_k}_l &= {w^p}^\top \tanh(W^{pm} M_l^{p_k} + W^{ps} s_t +b^p), \\
\alpha^p_t &= \mathrm{softmax}([e^{p_1}; \ldots; e^{p_K}]), \\
c^p_t &=  \textstyle \sum_{l} \alpha^p_{tl} M^{p_\mathrm{all}}_{l}, 
\end{align}
where $w^q$, $w^p \in \mathbb{R}^d$, 
$W^{qm}$, $W^{qs}$, $W^{pm}$, $W^{ps}  \in \mathbb{R}^{d \times d}$, and $b^q$, $b^p \in \mathbb{R}^d$ are learnable parameters.

$P^q$ and $P^p$ are the copy distributions over the extended vocabulary, defined as:
%\begingroup\makeatletter\def\f@size{9.5}\check@mathfonts
\begin{align}
P^q(y_t) &=  \textstyle \sum_{j: x^q_j = y_t} \alpha^q_{tj}, \\
P^p(y_t) &= \textstyle \sum_{l: x^{p_{k(l)}}_{l} = y_t} \alpha^p_{tl},
\end{align}
%\endgroup
where $k(l)$ means the passage index corresponding to the $l$-th word in the concatenated passages.

%, as shown in~Figure~\ref{fig:copy}


\section{Experiments}

\subsection{Setup}


\paragraph{Datasets and styles.}
We conducted experiments on the two tasks
of MS MARCO 2.1~\citep{Bajaj18}.


\begin{table}[t!]
\centering
{\small \tabcolsep=1.5pt %3.5pt
\begin{tabular}{p{15em}|c|cc}
\hline
Model & train & Rouge-L & Bleu-1 \\ \hline
Masque (NLG style; single)
%\footnote{It was trained with all data (ALL train set). } 
& ALL&  {\bf 69.77} & {\bf 65.56} \\ %\hline
\begin{tabular}{p{15em}}
w/o multi-style learning (\S\ref{sec:style})
%\footnote{It was trained with well-formed data (WFA train set).}
\end{tabular} & WFA &68.20 & 63.95 \\
\begin{tabular}{p{15em}}
%\hspace{1em}
$\hookrightarrow$
w/o Transformer (\S\ref{sec:tfenc}, \S\ref{sec:style})%\footnotemark[3] %WFA
\end{tabular} & WFA & 67.13 & 62.96 \\ 
\begin{tabular}{p{15em}}
w/o passage ranker (\S\ref{sec:ranker})%\footnotemark[3] % WFA
\end{tabular} & WFA & 68.05 & 63.82 \\
\begin{tabular}{p{15em}}
w/o possibility classifier (\S\ref{sec:classifier})
%\footnote{It was trained with answerable data (ANS train set).}
\end{tabular} & ANS & 69.64 & 65.41 \\ \hline
Masque w/ gold passage ranker & ALL & 78.70 & 78.14 \\ 
\hline
\end{tabular}
}
\caption{RC performance of our models for Rouge-L and Bleu-1 on the WFA dev.~set. The models were trained with the dataset described in the column 'train'.}
\label{tb:ablation}
\end{table}

\paragraph{Does our multi-style learning improve NLG performance?}

Multi-style learning allowed our model to improve NLG performance by also using non-sentence answers.


%DuoRC~\citep{KhapraSSA18} is a dataset about movie strories, where QAs are created from different versions of a document narrating the same underlying story.  Most of the answers are very short (a sinlge word or a short phrase) \textcolor{red}{CoQA~\citep{ReddyCM18}} contains multi-turn QAs obtained from conversations about text passages. The questions and answers are shorter than those of SQuAD due to the nature of conversatinal QA.


\section{Conclusion}
We believe our study makes two contributions to the study of multi-passage RC with NLG.


%\fontsize{9.0pt}{10.0pt} \selectfont
\bibliographystyle{acl_natbib_nourl}
\bibliography{references}

%%%%%%%%%%%%%%%%%%%%%%%%%%%%%%%%%%%%5

\newpage
\appendix
\onecolumn
\section{Reading Comprehension Examples generated by Masque from MS MARCO 2.1}
\label{sec:examples}

\begin{table*}[h!]
\centering
\caption{Our model could control answer styles appropriately for (a) natural language, (b) cloze-style, and (c) keywords questions. (d) The Q\&A was incorrect. (e) %Passage re-ranking worked poorly and 
The answers were not consistent between the styles. (f) Copying from numerical words worked poorly. There were some \underline{grammatical errors}.  }
\label{tb:examples}
{\footnotesize
\tabcolsep=1pt
\vspace{0.5pt}
\begin{tabular}{p{50em}}
\hline
\vspace{0.5pt}
\pbox{50em}{
\textbf{(a) Question}: why your body would feel like it is shaking\\
\textbf{Relevant Passage}: Shaking is a symptom in which a person has tremors (shakiness or small back and forth movements) in part or all of his body. Shaking can be due to cold body temperatures, rising fever (such as with infections), neurological problems, medicine effects, drug abuse, etc. ...Read more. \\
%\textbf{Passage Rank}: 1 \\
\textbf{Reference Answer (Q\&A)}: Shaking can be due to cold body temperatures, rising fever (such as with infections), neurological problems, medicine effects, drug abuse, etc.  \\
\textbf{Prediction (Q\&A)}: because of cold body temperatures , rising fever , neurological problems , medicine effects , drug abuse~.~\cmark\\
\textbf{Reference Answers (NLG)}: Shaking can be due to cold body temperatures, rising fever, neurological problems, medicine effects and drug abuse. / 
Body would feel like it is shaking due to cold body temperatures, rising fever, neurological problems, medicine effects, drug abuse. \\\
\textbf{Prediction (NLG)}: your body would feel like it is shaking because of cold body temperatures , rising fever , neurological problems , medicine effects , drug abuse . \cmark
}
\vspace{1pt}
\\ \hline
\vspace{0.5pt}
\pbox{50em}{
\textbf{(b) Question}: \_\_\_\_\_ is the name used to refer to the era of legalized segregation in the united states \\
\textbf{Relevant Passage}: Jim Crow law, in U.S. history, any of the laws that enforced racial segregation in the South between the end of Reconstruction in 1877 and the beginning of the civil rights movement in the 1950s.
\\
%\textbf{Passage Rank}: 3 \\
\textbf{Reference Answer (Q\&A)}: Jim Crow \\
\textbf{Prediction (Q\&A)}: jim crow \cmark \\
\textbf{Reference Answer (NLG)}: Jim Crow is the name used to refer to the era of legalized segregation in the United States. \\
\textbf{Prediction (NLG)}: jim crow is the name used to refer to the era of legalized segregation in the united states . \cmark
}
\vspace{1pt}
\\ \hline
\vspace{0.5pt}
\pbox{50em}{
\textbf{(c) Question}: average height nba player\\
\textbf{Relevant Passage}: The average height of an NBA player is around 6 feet 7 inches. The tallest NBA player ever was Gheorghe Muresan, who was 7 feet 7 inches tall. In contrast, the shortest NBA player ever was Tyrone Muggsy Bogues, who was 5 feet 3 inches tall. \\
%\textbf{Passage Rank}: 1 \\
\textbf{Reference Answer (Q\&A)}: Around 6 feet 7 inches \\
\textbf{Prediction (Q\&A)}: 6 feet 7 inches	\cmark \\
\textbf{Reference Answers (NLG)}: The average height of NBA players is around 6 feet, 7 inches. / The height of NBA player is around 6 feet 7 inches.\\
\textbf{Prediction (NLG)}: the average height of \underline{an} national basketball association player is 6 feet 7 inches . \cmark
}
\vspace{1pt}
\\ \hline 
\vspace{0.5pt}
\pbox{50em}{
\textbf{(d) Question}: population of henryetta ok\\
\textbf{Relevant Passage}: With it's population of 5,927 people, Henryetta represented 14.79\% of the county's total. In addition, where the city of Henryetta has a population density of 891 people per square mile [344.2 people/km2], Okmulgee County averages 57.5 people per square mile [22.19 people/km2].\\
%\textbf{Passage Rank}: 1 \\
\textbf{Reference Answer (Q\&A)}: 5927 \\
\textbf{Prediction (Q\&A)}: the population of henryetta , oklahoma is 5,927 .   \cmark \hspace{.5em} (content is OK, but style is NG \xmark)\\
\textbf{Reference Answer (NLG)}: The population of Henryetta, Oklahoma is 5,927. \\
\textbf{Prediction (NLG)}:  the population of henryetta , oklahoma is 5,927 . \cmark
}
\vspace{1pt}
\\ \hline 
\vspace{0.5pt}
\pbox{50em}{
\textbf{(e) Question}: does gameplay programmer need math skill\\
%\textbf{Passage Rank}: 8 \\
\textbf{Relevant Passage}: A good computer programmer is more of a problem solver and logical thinker than a math buff. And besides, the industry is peppered with many successful computer programmers who do not really know much about mathematics. \\
\textbf{Reference Answer (Q\&A)}: No \\
\textbf{Prediction (Q\&A)}: yes \xmark \\
\textbf{Reference Answers (NLG)}: No, a gameplay programmer doesn't need math skill. / No, gameplay programmer do not need an math skill.\\
\textbf{Prediction (NLG)}: no , \underline{gameplay programmer} does not need math skill . \cmark
}
\vspace{1pt}
\\ \hline
\vspace{0.5pt}
\pbox{50em}{
\textbf{(f) Question}: how long does a freezer take to cool down\\
\textbf{Relevant Passage}: Quick Answer. It takes anywhere from three to 24 hours for a refrigerator to reach safe temperatures for storing food, depending on the size and type of unit. When the refrigerator compartment reaches 40 degrees Fahrenheit and the freezer reaches 5 degrees Fahrenheit, it is safe to transfer food items. Keep Learning. \\
%\textbf{Passage Rank}: 3 \\
\textbf{Reference Answer (Q\&A)}: 24 hours\\
\textbf{Prediction (Q\&A)}: 4 to 5 hours \xmark \\
\textbf{Reference Answers (NLG)}: A freezer takes 24 hours to cool down. / A  freezer take to cool down is 24 hours.\\
\textbf{Prediction (NLG)}: a freezer takes 4 to 12 hours to cool down . \xmark
}
\vspace{1pt}
\\ \hline
\end{tabular}
}
\end{table*}

\end{document}
